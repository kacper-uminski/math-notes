\documentclass[a4paper,12pt]{article}
\setlength{\parindent}{0pt}
\setlength{\parskip}{1.5ex} 

\usepackage{amsmath}
\usepackage{amssymb}
\newcommand{\partialfrac}[2]{\frac{\partial #1}{\partial #2}}


\usepackage[includeheadfoot,left=2cm,right=2cm,top=2cm,bottom=2cm]{geometry}

\usepackage[utf8]{inputenc}



\title{TATA76 - Föreläsningsanteckningar}
\author{Kacper Uminski}
\date{}

\begin{document}
\begin{titlepage}
  \clearpage
  \maketitle
  \thispagestyle{empty}
\end{titlepage}
\section{Rummet $\mathbb{R}^n$, Grundbegrepp, och Funktioner av Flera Variabler}
\section{Gränsvärden och Kontinuitet}
\section{Differentierbarhet och Partiella Derivator}
Kom ihåg från envariabelanalysen att
$f'(a) = lim_{h \to 0}\frac{f(a+h)-f(a)}{h}$.

$f'(a)$ är lutningen hos tangenten i punkten $(a,f(a))$
\subsection{Flera Variabler}
Låt $z=f(x,y)$. Fixera $y=b$, det vill säga studera $z=f(x,b)$. När x varierar
beskriver detta samband en kurva och vi definierar
\begin{equation}
  f'_x(a,b) = lim_{h \to 0}\frac{f(a+h,b)-f(a,b)}{h}
\end{equation}
$f'_x(a,b)$ blir lutningen i $x$-led i punkten (a,b,f(a,b)).

Detta betecknas $f'_x, \partialfrac{f}{x}, D_xf$. Krokiga $\partial$
markerar att f beror av \textit{flera} variabler. Raka $d$ markerar att $f$
beror av \textit{en} variabel.

På motsvarande sätt, om vi fixerar $x=a$ och låter $y$ variera, definieras
\begin{equation*}
  f'_y(a,b) = lim_{k \to 0}\frac{f(a,b+k)-f(a,b)}{k}
\end{equation*}
Detta betecknas $f'_y, \partialfrac{f}{y}, D_yf$, och beskriver
lutningen i $y$-led.

Av definitionen framgår att räkneregler för derivator ser ut som förr. Deriverar
man med avseende på en variabel, så är alla andra variabler konstanta.

\paragraph{Ex.1:}
  \begin{equation*}
    f(x,y,z) = xy^2z^3 \Rightarrow 
    \begin{cases}
      \partialfrac{f}{x}=y^2z^3\\ \\
      \partialfrac{f}{y}=2xyz^3\\ \\
      \partialfrac{f}{z}=3xy^2z^2
    \end{cases}
  \end{equation*}

\paragraph{Ex.2:}
  \begin{equation*}
    f(x,y)=ye^{xy}+\sin(x^2+2y) \Rightarrow
    \begin{cases}
      \partialfrac{f}{x} = y^2e^{xy}+2x\cos(x^2+2y) \\ \\
      \partialfrac{f}{y} = (1+xy)e^{xy}+2\cos(x^2+2y)
    \end{cases}\\
  \end{equation*}

\textbf{OBS:}
  Om f är en funktion av två variabler, så har grafen till f inte \textit{en}
  lutning. Lutningen beror på åt vilket håll man tittar.

\paragraph{Def:}
  Om de partiella derivatorna $(f'_x, f'_y, etc)$ är kontinuerliga så sägs $f$
  vara av klass $C^1$.

\paragraph{Påstående:}
  Alla elementära funktioner och sammansättningar, summor, produkter och kvoter
  av sådana är kontinuerliga (och de är av klass $C^1,C^2,C^3,...$).

\textbf{OBS:}
  I envariabelanalysen gäller att om $f$ är deriverbar så är $f$ kontinuerlig.
  Detta gäller \textit{inte} i flervariabelanalys.

\paragraph{Ex.3:}
  \begin{equation*}
    f(x,y) =
    \begin{cases}
      \frac{xy}{x^2+y^2}, (x,y)\neq(0,0) \\
      0, (x,y)=(0,0)
    \end{cases}
  \end{equation*}

\section{Kedjeregeln och Partiella Differentialekvationer}
\subsection{Kedjeregeln, en variabel}
\paragraph{Kom ihåg:} $D(f(g(x))) = f'(g(x)) \cdot g'(x)$, eller med $t = g(x)$
så fås $f(g(x)) = f(t)$, där $t = g(x)$ och $\frac{df}{dx} = \frac{df}{dt} \cdot
\frac{dt}{dx}$.

\paragraph{Ex.1:}
Betrakta Eulerekvationen $x^2y''-2xy'+2y = 2x^2,x>0$: Byt variabel, $x = e^t$,
det vill säga $t = \ln x$, så fås:
\begin{equation*}
  \frac{dy}{dx} = \frac{dy}{dt} \cdot \frac{dt}{dx} =
  \frac{1}{x} \cdot \frac{dy}{dt}
\Leftrightarrow
  \frac{d}{dx} = \frac{1}{x} \cdot \frac{d}{dt}
\end{equation*}

\begin{align*}
  & \frac{d^2y}{dx^2} \\
  & = \frac{d}{dx}\left(\frac{dy}{dx}\right) \\
  & = \frac{d}{dx}\left(\frac{1}{x} \cdot \frac{dy}{dt}\right) \\
  & \stackrel{produktregeln}{=} -\frac{1}{x^2} \cdot \frac{dy}{dt}
    + \frac{1}{x} \cdot \frac{d}{dx}\left(\frac{dy}{dt}\right) \\
  & = \frac{1}{x^2}\left(\frac{d^2y}{dt^2} - \frac{dy}{dt}\right)
\end{align*}
Detta ger den nya ekvationen
\begin{align*}
  VL & = x^2\frac{d^2y}{dx^2}-2x\frac{dy}{dx}+2y \\
     & = ... \\
     & = \frac{d^2y}{dt^2}-3\frac{dy}{dt}+2y \\
     & = 2x^2 \stackrel{x = e^t}{=} 2e^{2t}
\end{align*}
Det vill säga, $\frac{d^2y}{dt^2} - 3\frac{dy}{dt}+2y = 2e^{2t}$, som har
lösningarna $y = Ae^t+Be^{2t}-2te^{2t} = Ax+Bx^2-2x^2\ln x$.

\subsection{Kedjeregeln i flera variabler}
Vi förbereder med att titta på tangentplan.

Tangentplanets ekvation:
\begin{equation*}
  z = f(a,b) + f'_x(a,b)\cdot (x-a) + f'_y(a,b)\cdot (y-b)
\end{equation*}

Jämför med tangentens ekvation i envariabelanalys:
\begin{equation*}
  y = f(a)+f'(a)\cdot(x-a)
\end{equation*}

Med $h = \Delta x$ och $k = \Delta y$, och med
$\Delta z = f(a+\Delta x, b+\Delta y) - f(a,b)$ fås
$\Delta z \approx \partialfrac{f}{x}\frac{\Delta x}{\Delta t}
+ \partialfrac{f}{x} \cdot \frac{\Delta y}{\Delta t}$. Om $\Delta t$ är litet
och om $f \in C^1$. Låter vi $\Delta t \to 0$ så kan man visa att:

\begin{equation*}
  \frac{dz}{dt} = \partialfrac{f}{x} \cdot \frac{dx}{dt}
  + \partialfrac{f}{y} \cdot \frac{dy}{dt} \leftarrow \text{Kedjeregeln}
\end{equation*}

Alltså: Om $z = f(x,y)$, där $x=x(t), y=y(t)$, så är
$\frac{dz}{dt} = \partialfrac{f}{x} \cdot \frac{dx}{dt} +
\partialfrac{f}{y} \cdot \frac{dy}{dt}$, eller med $f=f(x,y)$ så fås
$\frac{df}{dt} = \partialfrac{f}{x} \cdot \frac{dx}{dt} 
\partialfrac{f}{y} \cdot \frac{dy}{dt}$.

\paragraph{Ex.2:} Låt $f(u,v) = v\sin(uv)$, där $u=x^2, v=3x$. Då är
$f = 3x\sin(3x^3)$. Kedjeregeln ger $\frac{df}{dx}=$

\section{Tangentplan, Gradient, och Riktningsderivata}
\subsection{Gradient}
Om $f=f(x,y)$ har kontinuerliga partiella derivator (eg: f differentierbar)
så definieras $\nabla f$ (gradienten av $f$) som
\begin{equation*}
  \nabla f = \begin{pmatrix}
               f'_x\\
               f'_y
             \end{pmatrix}
\end{equation*}

\paragraph{Ex.1}
  \begin{equation*}
    f(x,y,z) = ln(x^2+y)+e^{yz} \Rightarrow \nabla f =
    \begin{pmatrix}
      \partialfrac{f}{x}\\\\
      \partialfrac{f}{y}\\\\
      \partialfrac{f}{z}
    \end{pmatrix}
  \end{equation*}
  OBS: $\nabla f$ är en vektor.

\subsection{Tangent till en kurva}
Låt $\Gamma$ vara en kurva i planet (eller rummet), som ges av
$(x,y) = (x(t),y(t)) = \vec{\phi}(t)$. En tangent till $\Gamma$ söks.
Titta i $t$ och i $t+h$:
\begin{equation*}
  lim_{h \to 0}\frac{\vec{\phi}(t+h)-\vec{\phi}(t)}{h}
\end{equation*}
Om gränsvärdet finns, $\neq\vec{0}$; så kommer vi att få tangentvektor till
kurvan, $\vec{\phi}'(t) = (x'(t),y'(t))$.

\paragraph{Ex.2:}
  Bestäm ekvationen för linjen som tangerar kurvan $(x,y,z) = (t, t^2, t^3)$ i
  punkten där $t = -1$.

  \textbf{Lösning:} $t = -1$ ger $(x,y,z) = (-1,1,-1)$ En tangentvektor fås som
  \begin{equation*}
    \begin{pmatrix}
      x'(t)\\
      y'(t)\\
      z'(t)
    \end{pmatrix}
    =
    \begin{pmatrix}
      1 \\
      2t \\
      3t^2
    \end{pmatrix}
    \stackrel{t = -1}{=}
    \begin{pmatrix}
      1 \\
      -2 \\
      -3
    \end{pmatrix}
  \end{equation*}

  Så tangentens ekvation är

  \begin{equation*}
    \begin{pmatrix}
      x \\
      y \\
      z
    \end{pmatrix}
    =
    \begin{pmatrix}
      -1 \\
      1 \\
      -1
    \end{pmatrix}
    + t
    \begin{pmatrix}
      1 \\
      -2 \\
      3
    \end{pmatrix}, t \in \mathbb{R}
  \end{equation*}

  Betrakta en nivåyta, $f(x,y,z) = C$ och låt $\Gamma$ vaa en kurva på denna
  yta. $\Gamma$ ges av $(x,y,z)=(x(t),y(t),z(t))$. Längs denna kurva är alltså
  $f(x(t), y(t), z(t)) = C$. Derivera med avseende på $t$, så fås
  $\frac{df}{dt}=0$ men
  \begin{equation*}
    \frac{df}{dt} =
    \partialfrac{f}{x} \frac{dx}{dt} +
    \partialfrac{f}{y} \frac{dy}{dt} +
    \partialfrac{f}{z} \frac{dz}{dt} =
    \begin{pmatrix}
      \partialfrac{f}{x} \\\\
      \partialfrac{f}{y} \\\\
      \partialfrac{f}{z}
    \end{pmatrix}
    \cdot
    \begin{pmatrix}
      \frac{dx}{dt} \\\\
      \frac{dy}{dt} \\\\
      \frac{dz}{dt}
    \end{pmatrix}
    = \nabla f \cdot \vec{\phi}'(t) = 0
  \end{equation*}
  Men $\overline{\phi}$ ligger i ytan, så $\vec{\phi}'$ är tangentvektor
  till ytan och detta gäller alla kurvor i ytan. Alltså är $\nabla f$ vinkelrät
  mot alla vektorer som tangerar ytan, då $\nabla f$ är en normalvektor till
  nivåytan $f = C$.

\paragraph{Ex.3:}
  Sök tangentplanet till ytan $z = x^2+y^2$ i punkten $(-1,1)$. $x=-1, y=1$ ger
  $z=2$. Skriv om ytan som nivåyta.
  \begin{equation*}
    z=x^2+y^2 \Leftrightarrow F(x,y,z) = x^2+y^2-z = 0
    \Rightarrow \nabla F =
    \begin{pmatrix}
      2x \\
      2y \\
      -1
    \end{pmatrix}
  \end{equation*}

  Så $\nabla F$ är en normal till ytan (i varje punkt på ytan) speciellt i givna
  punkten är $(x,y,z)=(-1,1,2)$ så $\nabla F = (-2, 2, -1)$ och $(x,y,z)$ ligger
  i planet omm \\$(-2,2,-1) \cdot (x+1, y-1, z-2) = 0$ det vill säga
  $-2x+2y-z = 2$.

\subsection{Riktningsderivata}
Låt $z = f(x,y)$. Studera $z$ då $(x,y) = (a_1,a_2)+t(v_1,v_2)$ (beskriver en
linje i planet, genom $(a_1,a_2)$, riktning $(v_1, v_2)$) där
$|\vec{v}| = |(v_1, v_2)| = 1$. Då definieras
\begin{equation*}
  f'_{\vec{v}}(\vec{a}) =
  lim_{t \to 0}\frac{f(\vec{a}+t\vec{v})-f(\vec{a})}{t}
\end{equation*} om gränsvärdet finns.

\textbf{OBS:}
  Om $\vec{v} = \hat{x}$ så är $f'_{\vec{v}}(\vec{a}) = f'_x(\vec{a})$. Samma
  sak gäller i $y$-led.

\textbf{OBS:}
  Låt $h(t) = f(\vec{a}+t\vec{v}) = f(\vec{g}(t))$, där
  $\vec{g}(t)=\vec{a}+t\vec{v}$. Då fås

  \begin{equation*}
    \frac{\vec{f}(\vec{a}+t\vec{v})-f(\vec{a})}{t} = \frac{h(t)-h(0)}{t}
    \to h'(0), t \to \vec{0}
  \end{equation*}
  men
  \begin{align*}
    h'(0) \stackrel{kedjeregeln}{= } & \nabla f(\vec{g}(0)) \cdot \vec{g}'(0) \\
    = & \nabla f(\vec{a}) \cdot \vec{g}'(0) \\
    = & \nabla f(\vec{a}) \cdot \vec{v}, |\vec{v}|=1
  \end{align*}

  Således blir $f'_{\vec{v}} = \nabla f(\vec{a})\cdot \hat{v}$ om $f$
  differentierbar.

\textbf{OBS:}
  $f'_{\vec{v}}(\vec{a})=\nabla f(\vec{a}) \cdot \hat{v} =
  |\nabla f(\vec{a})||\hat{v}|\cos\alpha$, där $\cos\alpha$ antar alla värden i
  $[-1,1]$, så max ($|\vec{v}|=1$) $f'_{\vec{v}}(\vec{a})=|\nabla f(\vec{a})|$
  när $\alpha = 0$ och min $f'_{\vec{v}}(\vec{a})=-|\nabla f(\vec{a})|$ när
  $\alpha = -\pi$.

\textbf{Alltså:}
  Om $z=f(x,y)$ är en funktionsyta så pekar $\nabla f$ ut den riktning i
  $xy$-planet, där funktionen växer snabbast.

Vi har två tolkningar av gradient:
\begin{itemize}
\item Funktionsyta, $z = f(x,y)$. $\nabla f(\vec{a})$ pekar ut riktning i
  $xy$-planet där funktionen växer fortast.
\item Nivåyta, $F(x,y,z)=C$. $\nabla F(a,b,c)$ är normal till ytan.
\end{itemize}

\section{Kurvor, Ytor, och Funktionsdeterminanter}
En kurva på parameterform har utseendet $(x,y,z) = (x(t),y(t),z(t))$. På
motsvarande sätt beskriver $\vec{r}(s,t) = (x(s,t), y(s,t), z(s,t))$
en yta i rummet. Fixt $s = s_0$ ger kurvan
$\vec{r}(s_0,t) = (x(s_0,t), y(s_0,t), z(s_0,t))$ som ligger på ytan.
Denna kurvan har tangentvektorn $\partialfrac{\vec{r}}{t}(s_0,t)$.
Fixt $t = t_0$ ger på samma sätt en kurva på ytan, med tangentvektorn
$\partialfrac{\vec{r}}{s}(s,t_0)$. En normalvektor till ytan, i punkten
$(s_0,t_0)$, fås av:
\begin{equation*}
  \vec{n} =
  \partialfrac{\vec{r}}{s}(s_0,t_0) \times
  \partialfrac{\vec{r}}{t}(s_0,t_0)
\end{equation*}

\paragraph{Ex.1:} Bestäm tangentplanet till ytan
  \begin{equation*}
    \begin{cases}
      x = 2\sin\Theta \cos\phi \\
      y = 3\sin\Theta \sin\phi \\
      z = 4\cos\Theta \\
      \phi \in [0,2\pi] \\
      \Theta \in [0,\pi]
    \end{cases}
  \end{equation*}
  i punkten $\vec{r} = (x,y,z) = (0,3,0)$.

  \textbf{Lösning:} Vi bestämmer $\phi$ och $\Theta$ i denna punkt.

\section{Dubbelintegraler}
\section{Variabelbyte i dubbelintegraler}
\section{Trippelintegraler}
\section{Integraltillämpningar}
\end{document}

